\documentclass[11pt]{article}
\usepackage[T1]{fontenc}
\usepackage[english]{babel}
\usepackage{graphicx}
\usepackage{palatino}
\usepackage{helvet}
\usepackage{times}
\usepackage{layout}
\usepackage[a4paper,top=2.0cm, right=2.0cm, bottom=2.0cm, left=2.0cm]{geometry}
\usepackage{enumitem}
\usepackage{amsthm}
\usepackage{url}
\usepackage{multicol,caption}
\usepackage{amsmath}
\usepackage{amssymb}

\setlist{nolistsep}

\usepackage{fancyhdr}
\pagestyle{fancy}
\lhead{{\hvnb MathSys CDT}}
\chead{}
\rhead{}
\lfoot{}
\cfoot{}
\rfoot{}

\newcommand*{\helvetica}{\fontfamily{phv}\selectfont}
\newcommand*{\helveticanarrow}{\fontfamily{phv}\fontseries{mc}\selectfont}
\newcommand*{\hvnb}{\fontfamily{phv}\fontseries{bc}\selectfont}
\newcommand*{\palatino}{\fontfamily{ppl}\selectfont}
\newcommand*{\timesroman}{\fontfamily{ptm}\selectfont}

% Commands to produce formatted layout
\newcommand*{\projecttitle}[1]{\begin{center}\Large\hvnb{\color{blue} #1}\end{center}}
\newcommand*{\supervisor}[1]{\noindent \hvnb{\color{black}Lecturer:} \helveticanarrow{#1}\\}
\newcommand*{\R}{\mathbb{R}}
\newcommand*{\xv}{\mathbf{x}}
\newcommand*{\yv}{\mathbf{y}}
\newcommand*{\cv}{\mathbf{c}}
\newcommand*{\bv}{\mathbf{b}}
\newcommand*{\av}{\mathbf{a}}
\newcommand*{\dv}{\mathbf{d}}

% Support for multiple bibliographies
\usepackage[sectionbib,numbers]{natbib}
\usepackage{chapterbib}

\usepackage[compact]{titlesec}
\usepackage{lipsum}
\usepackage{xcolor}


\titleformat{\section}
  {\large\hvnb\color{blue}}{\thesection}{1em}{}
\titleformat{\subsection}
  {\hvnb}{\thesubsection}{1em}{}
\titleformat{\chapter}
  {\Large\hvnb\color{blue}}{Lecture \thechapter:}{1em}{}

\newenvironment{Figure} 
  {\par\medskip\noindent\minipage{\linewidth}}
  {\endminipage\par\medskip}


\begin{document}
\setlength{\bibsep}{0.2pt}
\setlength{\itemsep}{0.2pt}

\helveticanarrow
\rhead{\hvnb MA934 - Numerical algorithms and optimisation}

\projecttitle{Sample viva questions and topics}

\supervisor{Colm Connaughton} 



\begin{multicols}{2}
\section*{List of examinable topics covered in 2019}
\begin{itemize}
\item Floating point arithmetic, truncation error and loss of significance.
\item Amplification of truncation error by instability
\item Iteration, recursion and memoization, divide and conquer paradigm.
\item Computational complexity of recursive algorithms
\item Solving linear recursion relations
\item Sorting: insertion sort, Shellsort and mergesort
\item Structural recursion: linked lists and binary trees. Basic idea of how to represent recursive data strunctures in Julia.
\item Binary tree search
\item Interval membership and Fenwick trees - connection to Gillespie algorithm for stochastic simulation
\item Bracket-and-bisect method for root finding.
\item Derivation and properties of Newton-Raphson method for root finding, convergence properties.
\item Newton-Raphson method in $\mathbb{R}^n$.
\item Convex functions, convex sets and convex optimisation
\item Linear programming: infeasible, unbounded and solvable problems, fundamental theorem of linear programming
\item Writing a linear programme in standard form, slack variables, basic feasible vectors.
\item Idea of Dantzig Simplex Algorithm (but {\bf not} detailed calculations).
\item Ideas of golden section search for one-dimensional minimisation (but {\bf not} detailed derivation.)
\item Method of steepest descent for unconstrained optimisation in $\mathbb{R}^n$.
\item Stochastic gradient descent.
\item Dual numbers and automatic differentiation.
\item Taylor's theorem and derivation of finite difference formulae for numerical approximation of derivatives.
\item Properties of Fourier transforms: transform of real functions, shift property, convolution theorem, Wiener-Kinchin theorem  
\item Sampling of functions, statement of the sampling theorem, Nyquist frequency, aliasing problem
\item Derivation of discrete Fourier Transform
\item Danielson-Lanczos lemma and Fast Fourier Transform algorithm.
\end{itemize}

\section*{Sample viva questions}
\begin{enumerate}
\item Explain the concept of machine precision and how loss of significance occurs in floating point arithmetic.
\item
Solve recurrence relations like the following:
\begin{displaymath}
a_n = a_{n-1} + a_{n-2}
\end{displaymath}
with $a_0=1$ and $a_1=1$.
\item
Write down an example of a recursive function (other than the factorial function!) and explain how it works. Explain the idea of a "divide-and-conquer" algorithm and give an example.
\item
Explain how the Shellsort algorithm works and why it is faster than insertion sort.
\item
Explain how the mergesort algorithm works and write down an equation for its computational complexity
\item
Consider a recurrence relation like the following for the computational complexity of a hypothetical divide-and-conquer algorithm:
\begin{displaymath}
F(n) = 2 F\left(\frac{n}{2}\right) +n
\end{displaymath}
with $F(1)=1$. Explain how to solve this recursion.
\item
What is meant by structural recursion and give some examples.
\item
Explain what is a linked list and describe how it compares to a linear array for storing a sequence of objects. Describe how to create a linked list in Julia.
\item
Explain what is a binary tree and outline how it can be used to perform search on a set of key-value pairs in $O(\log n)$ time where $n$ is the number of elements in the set to be searched.
\item
What is a Fenwick tree and how does it differ from a binary search tree? Explain how a Fenwick tree can be used to solve the interval membership problem efficiently. 
\item
Explain what it means for an interval $(a,b)$ to bracket a root of a function $f(x)$ of a single variable and explain how the bracket-and-bisect algorithm works.
\item
Derive the Newton Raphson method for finding roots of a function of a single variable and explain its advantages and disadvantages.
\item
Consider a set of $n$ nonlinear equations in $\mathbb{R}^n$:
\begin{displaymath}
\mathbf{F}(\xv) = 0.
\end{displaymath}
Derive the Newton Raphson method for multi-dimensional root finding. 
\item 
Explain the concepts of convex sets, convex functions and convex optimisation problems. Why is convexity so important in the theory of optimisation?
\item
What is a linear programme? Explain why the feasible set is a polygon in $\mathbb{R}^n$ and why the solution (if it exists) must be at a vertex of the feasible set.
\item
Explain how to put a general linear programme in standard form.
\item
Explain how Dantzig's simplex algorithm works.
\item
What does it mean for a triple $(a, b, c)$ to bracket a minimum of a function $f(x)$ of a single variable? Explain the golden section search algorithm to find a local minimum of a function of a single variable starting from a bracketing triple.
\item
Given a function, $f(\xv)$ of $n$ variables, what is the gradient of $f$? Given a point $\xv \in \mathbb{R}^n$ and a direction $\dv \in \mathbb{R}^n$, what is the line minimiser of $f(\xv)$ from $\xv$ in the direction $\dv$? Explain how the Method of Steepest Descent works.
\item
Describe the stochastic gradient descent algorithm and explain what types of optimisation problems it is intended to solve.
\item
Write down the addition, multiplication and conjugation rules for dual numbers. Explain how dual arithmetic provides automatic differentiation of functions of a single variable. 
\item
Derive the convolution theorem and explain why it is important.
\item
Define the power spectrum and autocorrelation function of a function, $f(t)$, and explain what they mean. Show that the autocorrelation function is the Fourier transform of the power spectrum.
\item
What is the Nyquist frequency in discrete signal processing? Why is it important? Explain the aliasing problem.
\item
Outline how the discrete Fourier transform is obtained from its continuous version.
\item
Describe how the divide-and-conquer approach to the discrete Fourier transform leads to the Fast Fourier Transform algorithm.
\end{enumerate}

\end{multicols}
\end{document}
